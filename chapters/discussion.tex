\chapter{Discussion}
\label{chap:discussion}

Compared to a keypoint approach, acting at the level of segments offers several advantages even without making assumptions about perfect segmentation or of the presence of `objects' in the environment. Though, should segments be used to represent objects, they then have the potential to be simultaneously useful for place recognition and for other systems in autonomous robots such as tracking and predicting the dynamic behavior of objects. The development of the autoencoder submodule was partly intended to work towards this goal. Environment navigation at the human-level appears to depend on deep and complex understanding, and it as an open question as to what algorithms will be used to achieve similar performance.\\

During the development of SegMatch, it became evident that segmentation is crucial to the final performance of the algorithm. Unreliable and low quality segments severely limit the loop closure potential of SegMatch. We expect that the development of a fast segmentation model, producing useful segments reliably despite gaps and artifacts in the range data - independently of the environment - is possible, though it may require machine learning. According to the litterature \cite{semantic}, similar research is being pursued in the field of semantic image segmentation.\\

We were suprised by the ability of the RANSAC algorithm to find patterns in the data. In fact, it is possible that with very good quality segments and no description at all, RANSAC would be able to find enough patterns in the arrangement of segments to find loop closures. In this case, it would be possible that the segmentation and description step be fused into one, using the descriptive information present in range data to extract higher quality, pre-described segments.\\

Detection of erroneous loop closures was not pursued in depth, however it would increase the robustness of the algorithm. In particular, loop closures proposed by segmatch could be validated using the score of an ICP registration between source and target, or by looking at conflicts between overlapping segments, after applying the loop closure transformation.\\

We have not experimented with SegMatch on all possible types of environment, however any environment with distinct objects should intuitively be a good candidate for this algorithm. Meta parameter adaptation could also be performed to for example vary the scale at which SegMatch operates depending on the detected scale of the robot's surroundings.\\

The choice of the models used in the unsupervised-learning description module was very dependent on the current memory and computational limitations of computer hardware. In the near future, it is expected that the improvement of the possibilities of computer hardware will allow for deeper real-time models to be trained. In addition as LiDAR sensors become cheaper and increasingly common, training data will be more readily available. Thus, it is expected that the unsupervised-learning model's description capabilities will increase accordingly.
