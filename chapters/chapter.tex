\chapter{Einige wichtige Hinweise zum Arbeiten mit \LaTeX\ }
\label{sec:latexumg}

Nachfolgend wird die Codierung einiger oft verwendeten Elemente
kurz beschrieben. Das Einbinden von Bildern ist in \LaTeX\ nicht
ganz unproblematisch und hängt auch stark vom verwendeten Compiler
ab. Typisches Format für Bilder in \LaTeX\ ist
EPS\footnote{Encapsulated Postscript} oder PDF\footnote{Portable Document Format}.


\section{Gliederungen}
\label{sec:gliederung}

Ein Text kann mit den Befehlen \texttt{\textbackslash
chapter\{.\}}, \texttt{\textbackslash section\{.\}},
\texttt{\textbackslash subsection\{.\}} und \texttt{\textbackslash
subsubsection\{.\}} gegliedert werden.


\section{Referenzen und Verweise}
\label{sec:refverw}

Literaturreferenzen werden mit dem Befehl \texttt{\textbackslash
citep\{.\}} und \texttt{\textbackslash
citet\{.\}} erzeugt. Beispiele: ein Buch \citep{Raibert1986LeggedRobotsThatBalance}, ein Buch und ein Journal Paper \citep{Raibert1986LeggedRobotsThatBalance,Vukobratovic2004ZeroMomentPoint}, ein Konferenz Paper mit Erwähnung des Autors: \citet{Pratt1995SEA}.

Zur Erzeugung von Fussnoten wird der Befehl \texttt{\textbackslash
footnote\{.\}} verwendet. Auch hier ein Beispiel\footnote{Bla
bla.}.

Querverweise im Text werden mit \texttt{\textbackslash label\{.\}}
verankert und mit \texttt{\textbackslash cref\{.\}} erzeugt.
Beispiel einer Referenz auf das zweite Kapitel:
\cref{sec:latexumg}.


\section{Aufzählungen}\label{sec:aufz}

Folgendes Beispiel einer Aufzählung ohne Numerierung,
\begin{itemize}
  \item Punkt 1
  \item Punkt 2
\end{itemize}
wurde erzeugt mit:
\begin{verbatim}
\begin{itemize}
  \item Punkt 1
  \item Punkt 2
\end{itemize}
\end{verbatim}

Folgendes Beispiel einer Aufzählung mit Numerierung,
\begin{enumerate}
  \item Punkt 1
  \item Punkt 2
\end{enumerate}
wurde erzeugt mit:
\begin{verbatim}
\begin{enumerate}
  \item Punkt 1
  \item Punkt 2
\end{enumerate}
\end{verbatim}

Folgendes Beispiel einer Auflistung,
\begin{description}
  \item[P1] Punkt 1
  \item[P2] Punkt 2
\end{description}
wurde erzeugt mit:
\begin{verbatim}
\begin{description}
  \item[P1] Punkt 1
  \item[P2] Punkt 2
\end{description}
\end{verbatim}


\section{Erstellen einer Tabelle}\label{sec:tabellen}

Ein Beispiel einer Tabelle:
\begin{table}[h]
\begin{center}
 \caption{Daten der Fahrzyklen ECE, EUDC, NEFZ.}\vspace{1ex}
 \label{tab:tabnefz}
 \begin{tabular}{ll|ccc}
 \hline
 Kennzahl & Einheit & ECE & EUDC & NEFZ \\ \hline \hline
 Dauer & s & 780 & 400 & 1180 \\
 Distanz & km & 4.052 & 6.955 & 11.007 \\
 Durchschnittsgeschwindigkeit & km/h & 18.7 &  62.6 & 33.6 \\
 Leerlaufanteil & \% & 36 & 10 & 27 \\
 \hline
 \end{tabular}
\end{center}
\end{table}

Die Tabelle wurde erzeugt mit:
\begin{verbatim}
\begin{table}[h]
\begin{center}
 \caption{Daten der Fahrzyklen ECE, EUDC, NEFZ.}\vspace{1ex}
 \label{tab:tabnefz}
 \begin{tabular}{ll|ccc}
 \hline
 Kennzahl & Einheit & ECE & EUDC & NEFZ \\ \hline \hline
 Dauer & s & 780 & 400 & 1180 \\
 Distanz & km & 4.052 & 6.955 & 11.007 \\
 Durchschnittsgeschwindigkeit & km/h & 18.7 &  62.6 & 33.6 \\
 Leerlaufanteil & \% & 36 & 10 & 27 \\
 \hline
 \end{tabular}
\end{center}
\end{table}
\end{verbatim}


\section{Einbinden einer Grafik}\label{sec:epsgraph}

Das Einbinden von Graphiken kann wie folgt bewerkstelligt werden:
\begin{verbatim}
\begin{figure}
   \centering
   \includegraphics[width=0.75\textwidth]{images/k_surf.pdf}
   \caption{Ein Bild.}
   \label{fig:k_surf}
\end{figure}
\end{verbatim}

\begin{figure}
   \centering
   \includegraphics[width=0.75\textwidth]{images/k_surf.pdf}
   \caption{Ein Bild}
   \label{pics:k_surf}
\end{figure}

oder bei zwei Bildern nebeneinander mit:
\begin{verbatim}
\begin{figure}
  \begin{minipage}[t]{0.48\textwidth}
    \includegraphics[width = \textwidth]{images/cycle_we.pdf}
  \end{minipage}
  \hfill
  \begin{minipage}[t]{0.48\textwidth}
    \includegraphics[width = \textwidth]{images/cycle_ml.pdf}
  \end{minipage}
  \caption{Zwei Bilder nebeneinander.}
  \label{pics:cycle}
\end{figure}
\end{verbatim}

\begin{figure}
  \begin{minipage}[t]{0.48\textwidth}
    \includegraphics[width = \textwidth]{images/cycle_we.pdf}
  \end{minipage}
  \hfill
  \begin{minipage}[t]{0.48\textwidth}
    \includegraphics[width = \textwidth]{images/cycle_ml.pdf}
  \end{minipage}
  \caption{Zwei Bilder nebeneinander}
  \label{pics:cycle}
\end{figure}


\section{Mathematische Formeln}\label{sec:math}

Einfache mathematische Formeln werden mit der equation-Umgebung
erzeugt:
\begin{equation}
 p_{me0f}(T_e,\omega_e) \ = \ k_1(T_e) \cdot (k_2+k_3 S^2
 \omega_e^2) \cdot \Pi_{\mathrm{max}} \cdot \sqrt{\frac{k_4}{B}} \, .
 	\label{eq:my_equation}
\end{equation}

Der Code dazu lautet:
\begin{verbatim}
\begin{equation}
 p_{me0f}(T_e,\omega_e) \ = \ k_1(T_e) \cdot (k_2+k_3 S^2
 \omega_e^2) \cdot \Pi_{max} \cdot \sqrt{\frac{k_4}{B}} \, .
\end{equation}
\end{verbatim}

Mathematische Ausdrücke im Text werden mit \$formel\$ erzeugt (z.B.:
$a^2+b^2=c^2$).

Vektoren und Matrizen werden mit den Befehlen \texttt{\textbackslash vec\{.\}} und \texttt{\textbackslash mat\{.\}} erzeugt (z.B. $\vec{v}$, $\mat{M}$).


\section{Weitere nützliche Befehle}\label{sec:div}

Hervorhebungen im Text sehen so aus: \emph{hervorgehoben}. Erzeugt
werden sie mit dem \texttt{\textbackslash epmh\{.\}} Befehl.

Einheiten werden mit den Befehlen \texttt{\textbackslash unit[1]\{m\}} (z.B.~\unit[1]{m}) und \texttt{\textbackslash unitfrac[1]\{m\}\{s\}} (z.B.~\unitfrac[1]{m}{s}) gesetzt.
