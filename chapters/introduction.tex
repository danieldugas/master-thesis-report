\chapter{Introduction}
\label{sec:introduction}

This report is structured as follows: Chapter \ref{chap:segmatch} describes the implementations details of the segment-based loop closure algorithm which was developped during this project, and its usefulness and performance. The presented loop closure algorithm is from this point on referred to as SegMatch. SegMatch is also described and evaluated in Dubé \cite{segmatch}.\\

Chapter \ref{chap:ae} explores the development of a learning-based enhancement to SegMatch, allowing a module of the algorithm to learn useful segment descriptions in an unsupervised fashion. The enhancement is implemented and tested, and the results are discussed. The enhanced SegMatch algorithm, with autoencoder-based description is from here on out referred to as SegMatchAE.\\

\section{Motivation for Range-Based SLAM Loop Closures}
\label{sec:motivation}

When mapping the environment, in the absence of global positioning information, or other mechanisms to prevent errors from accumulating, drift will inevitably occur. Loop closures are such a mechanism of error correction, where recognition of previously visited environment allows knowledge of one's relative position to be updated.\\ 

This project examined specifically the implementation of such a method in the case where mapping is performed on data collected by a range-based sensor.\\

Range-based sensors allow for consistent depth-measurement, and can do so in environments where vision-based sensors become untrustworthy due to illumination or weather. Often, vision based sensors requires relatively complex algorithms to estimate depth. Range sensors can thus be necessary in scenarios where robust depth information is important.\\

When mapping the environment with a range-based sensor, the same algorithms as for vision-based mapping can not be used with similar success. A cause is the difference in the data produced by both sensors. As a result, even though algorithms for vision-based loop closures have been demonstrated, to our knowledge, no mature range-based equivalent exists.\\

\section{State-of-the-Art}
\label{sec:SOTA}

Parallel: Vision-based\\
Google Cartographer\\
Keypoint Matching\\

\section{Motivations for SegMatch}

Segments save space

The use of segments for place recognition has several useful properties.

\section{Motivations for SegMatchAE}

Generalization
Unsupervised
Less heuristic-based

